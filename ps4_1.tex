\documentclass[11pt,letterpaper]{article}
\usepackage[utf8]{inputenc}
\usepackage{amsmath}
\usepackage{amsfonts}
\usepackage{amssymb}
\usepackage[margin=1in]{geometry}
\usepackage{graphicx}
\usepackage{placeins}
\usepackage{listings}

\begin{document}
\noindent Landon Carter \\
9/22/16
6.111 PS4

\section{}
\subsection{A}
The smallest clock period is 9 ns. That's the $t_{PD}$ of the two inverters, plus $t_{PD}$ for the first D register, plus the $t_S$ of the second register.

\subsection{B}
The new circuit would not work correctly, because the $t_H$ of the second register would not be met.

\subsection{C}
S0 = 0, S1 = 0

\subsection{D}
S0 = 0, S1 = 1

\subsection{E}
The skew actually helps us - we can now have a clock period of 8 ns.

\section{}

\begin{lstlisting}[language=Verilog]
module jk
	(input j, k, preset, clear, clk,
	 output q, qbar);

	reg state;

	always @(posedge clk or negedge preset or negedge clear) begin
		if (preset && claer) begin  // standard operation
			case ({j, k})
				2'b00: state <= state;
				2'b01: state <= j;
				2'b10: state <= j;
				2'b11: state <= ~state;
				default: state <= state;
			endcase
		end
		else if (~preset) begin  // set state
			state <= 1;
		end
		else if (~clear) begin  // clear state
			state <= 0;
		end
	end

	assign q = state;
	assign qbar = ~q;

endmodule
\end{lstlisting}



\end{document}
